% arara: pdflatex: { action: nonstopmode }
% arara: biber
% arara: pdflatex: { action: nonstopmode }
% arara: pdflatex: { action: nonstopmode }
% --------------------------------------------------------------------------
% the ELEMENTS package
% 
%   retrieve different properties of chemical elements
% 
% --------------------------------------------------------------------------
% Clemens Niederberger
% Web:    http://www.mychemistry.eu/forums/forum/elements/
% E-Mail: contact@mychemistry.eu
% --------------------------------------------------------------------------
% Copyright 2015--2018 Clemens Niederberger
% 
% This work may be distributed and/or modified under the
% conditions of the LaTeX Project Public License, either version 1.3
% of this license or (at your option) any later version.
% The latest version of this license is in
%   http://www.latex-project.org/lppl.txt
% and version 1.3 or later is part of all distributions of LaTeX
% version 2005/12/01 or later.
% 
% This work has the LPPL maintenance status `maintained'.
% 
% The Current Maintainer of this work is Clemens Niederberger.
% --------------------------------------------------------------------------
% If you have any ideas, questions, suggestions or bugs to report, please
% feel free to contact me.
% --------------------------------------------------------------------------
\documentclass[load-preamble+]{cnltx-doc}
\usepackage[utf8]{inputenc}
\usepackage{elements}

\setcnltx{
  package = elements ,
  authors = Clemens Niederberger ,
  email   = contact@mychemistry.eu ,
  url     = http://www.mychemistry.eu/forums/forum/elements/ ,
  add-cmds = {
    atomicnumber,
    DeclareAtomIsotopes,
    DeclareAtomName,
    DeclareAtomSymbol,
    DeclareElectronDistribution,
    elconf,
    elementname,
    elementsymbol,
    mainelementisotope,
    printangularmomentum,
    saveatomicnumber,
    saveelementisotopes,
    saveelementname,
    saveelementsymbol,
    savemainelementisotope,
    setangularmomentum,
    setatomisotopes,
    setatomname,
    setatomsymbol,
    setelectrondistribution,
    writeelconf,
    Z
  } ,
  add-silent-cmds = {
    endhead, foo, ifnumless, LTleft, LTright, midrule, theelement, toprule,
    whileboolexpr
  } ,
  add-silent-envs = { longtable, landscape } ,
  abstract = {%
    This package provides means for retrieving properties of chemical elements
    like atomic number, element symbol, element name, electron distribution or
    isotope number.  Properties are defined for the elements up to the atomic
    number~112.

    This package is a spin-off of the package \pkg{bohr}~\cite{pkg:bohr} by
    the same author.
  } ,
  index-setup = {
     othercode = \footnotesize ,
     level = \addsec ,
     noclearpage
  } ,
  makeindex-setup = {
     columns = 3 ,
     columnsep = 1em
  }
}

\expandafter\def\csname libertine@figurestyle\endcsname{LF}
\usepackage[libertine]{newtxmath}
\expandafter\def\csname libertine@figurestyle\endcsname{OsF}

\usepackage{array,booktabs,longtable}
\usepackage{lscape}

\defbibheading{bibliography}{\addsec{References}}

\begin{document}

\section{Licence and Requirements}
\license

\elements\ loads the packages \pkg{etoolbox}~\cite{pkg:etoolbox} and
\pkg{translations}~\cite{pkg:translations}.

\section{Element Names}

\begin{commands}
  \command{elementname}[\Marg{\meta{atomic number}|\meta{element symbol}}]
    Prints the element name of a given element as defined with \cs{setatomname}.
  \command{setatomname}[\oarg{alt.\@ name}\marg{atomic number}\marg{element
    name}]
    Define or redefine the name of an element.  If \meta{element name}
    contains non-ascii symbols the optional argument \meta{alt.\@ name} must
    be given.  In this case \meta{alt.\@ name} must be used in \elements'
    other macros where an element's name can be given as argument.
  \command{DeclareAtomName}[\oarg{alt.\@ name}\marg{atomic
    number}\marg{element name}]
    This is the same as \cs{setatomname} but used before begin document or in
    packages/classes.
  \command{saveelementname}[\marg{cs}\Marg{\meta{atomic number}|\meta{element
      symbol}|\meta{element name}}]
    Saves the name of the given element as replacement text for the macro
    \meta{cs}.
\end{commands}

\begin{example}[side-by-side]
  \elementname{Cu} \par
  \elementname{11} \par
  \saveelementname\foo{28}
  \ttfamily\meaning\foo
\end{example}

\section{Element Symbols}

\begin{commands}
  \command{elementsymbol}[\Marg{\meta{atomic number}|\meta{element name}}]
    Prints the element symbol of a given element as defined with \cs{setatomsymbol}.
  \command{setatomsymbol}[\marg{atomic number}\marg{element symbol}]
    Define or redefine the symbol of an element.
  \command{DeclareAtomSymbol}[\marg{atomic number}\marg{element symbol}]
    This is the same as \cs{setatomsymbol} but used before begin document or in
    packages/classes.
  \command{saveelementsymbol}[\marg{cs}\Marg{\meta{atomic number}|\meta{element
      symbol}|\meta{element name}}]
    Saves the symbol of the given element as replacement text for the macro
    \meta{cs}.
\end{commands}

\begin{example}[side-by-side]
  \elementsymbol{13} \par
  \elementsymbol{Sulfur} \par
  \saveelementsymbol\foo{83}
  \ttfamily\meaning\foo
\end{example}

\section{Atomic Numbers}

\begin{commands}
  \command{atomicnumber}[\Marg{\meta{element symbol}|\meta{element name}}]
    Prints the atomic number of a given element.
  \command{Z}[\Marg{\meta{element symbol}|\meta{element name}}]
    An alias of \cs{atomicnumber} but only defined at begin document and only
    if it isn't defined already.
  \command{saveatomicnumber}[\marg{cs}\Marg{\meta{atomic number}|\meta{element
      symbol}|\meta{element name}}]
    Saves the atomic number of the given element as replacement text for the
    macro \meta{cs}.
\end{commands}

\begin{example}[side-by-side]
  \atomicnumber{U} \par
  \atomicnumber{Chlorine} \par
  \saveatomicnumber\foo{Kr}
  \ttfamily\meaning\foo
\end{example}

\section{Electron Configuration}

\begin{commands}
  \command{elconf}[\Marg{\meta{atomic number}|\meta{element
      symbol}|\meta{element name}}]
    Typesets the electron configuration of the given element.
  \command{writeelconf}[\marg{electron distribution}]
    Typesets the electron distribution \meta{electron distribution}.  The
    input is the same as described below for \cs{setelectrondistribution}.
  \command{setelectrondistribution}[\marg{atomic number}\marg{electron distribution}]
     This set the electron distribution associated with the atom number
     \meta{atomic number}.  \meta{electron distribution} is a comma-separated
     list of the number of electrons placed on each shell from inner to outer
     shell.  For example \verbcode+\setelectrondistribution{3}{2,0,1}+ would
     be an excited \elementname{3}.   The number of electrons with the same
     principal quantum number but a different angular quantum number are
     separated with a \code{+} ordered by the angular quantum number, \ie,
     first \emph{s}, then \emph{p}, then \emph{d}, and then \emph{f} .
     \elementname{29}’s distribution would be declared like this: \\
     \verbcode!\setelectrondistribution{29}{2,2+6,2+6+10,1}!. \\
     A declaration with \verbcode!\setelectrondistribution{29}{2,8,18,1}!
     would work but then \verbcode!\elconf{29}! would give the wrong results.
  \command{DeclareElectronDistribution}
    This is the same as \cs{setelectrondistribution} but used before begin
    document or in packages/classes.
  \command{printangularmomentum}[\marg{angular momentum}]
    \sinceversion{0.1e}This command is used be \cs{elconf} and
    \cs{writeelconf} for typesetting $\printangularmomentum{s}$,
    $\printangularmomentum{p}$, $\printangularmomentum{d}$, \ldots{} You can
    redefine it if you want to change how those functions are typeset (keeping
    in mind that it will be used in \emph{math mode}).  The initial definition
    is:\par
    \verbcode+\newcommand*\printangularmomentum[1]{\mathrm{#1}}+
\end{commands}

\begin{example}[side-by-side]
  \writeelconf{2,2+6,2+6+7} \par
  \elconf{Cl}
\end{example}

There is currently \emph{no way} to get the electron configuration in the
shortened way (\eg:
[Ar]$\printangularmomentum{d}^{10}4\printangularmomentum{s}^1$).

\section{Isotope Lists}

\begin{commands}
  \command{setatomisotopes}[\marg{atomic number}\marg{isotope list}]
    Defines or redefines the isotope list for a given element.  \meta{isotope
      list} should be a comma seperated list of integers.  One of the integers
    may be preceded with a \code{!} to mark the main isotope for the given
    element: \verbcode+\setatomisotopes{6}{10,11,!12,13,14,15,16}+
  \command{DeclareAtomIsotopes}[\marg{atomic number}\marg{isotope list}]
    This is the same as \cs{setatomisotopes} but used before begin document or
    in packages/classes.
  \command{saveelementisotopes}[\marg{cs}\Marg{\meta{atomic number}|\meta{element
      symbol}|\meta{element name}}]
    Saves the isotope list of the given element as replacement text for the
    macro \meta{cs}.
  \command{savemainelementisotope}[\marg{cs}\Marg{\meta{atomic number}|\meta{element
      symbol}|\meta{element name}}]
    Saves the main isotope of the given element as replacement text for the
    macro \meta{cs}.  If the isotope list of the element contains no main
    isotope \meta{cs} will be equivalent to \cs*{@empty}.
  \command{mainelementisotope}[\Marg{\meta{atomic number}|\meta{element
      symbol}|\meta{element name}}]
    Print\sinceversion{0.1d} the main isotope of the given element.  If the
    isotope list of the element contains no main isotope nothing will be
    printed.
\end{commands}

\begin{example}[side-by-side]
  \ttfamily
  \saveelementisotopes\foo{C}
  \meaning\foo\par
  \savemainelementisotope\foo{C}
  \meaning\foo\par
  \normalfont
  \mainelementisotope{C}
\end{example}

\clearpage

\addsec{Element Properties}

On the following pages a table containing the properties known to \elements\
is printed. For those interested: the code used to get the table is as follows
(using the packages \pkg{lscape}~\cite{pkg:lscape},
\pkg{longtable}~\cite{pkg:longtable} and \pkg{booktabs}~\cite{pkg:booktabs}).

\begin{sourcecode}[gobble=0]
\setlength\LTleft\fill
\setlength\LTright\fill
\newcounter{element}
\setcounter{element}{1}
\begin{landscape}
\begin{longtable}{lllll}
  \toprule
    Number & Symbol & Name & Main Isotope & Electron Configuration \\
  \midrule
  \endhead
  \whileboolexpr{test{\ifnumless{\value{element}}{113}}}
    {
      \theelement &
      \elementsymbol{\arabic{element}} &
      \elementname{\arabic{element}} &
      \mainelementisotope{\arabic{element}} &
      \elconf{\arabic{element}}
      \stepcounter{element} \\
    }
    {}
\end{longtable}
\end{landscape}
\end{sourcecode}

\setlength\LTleft\fill
\setlength\LTright\fill
\newcounter{element}
\setcounter{element}{1}
\begin{landscape}
\begin{longtable}{lllll}
  \toprule
    Number & Symbol & Name & Main Isotope & Electron Configuration \\
  \midrule
  \endhead
  \whileboolexpr{test{\ifnumless{\value{element}}{113}}}
    {
      \theelement &
      \elementsymbol{\arabic{element}} &
      \elementname{\arabic{element}} &
      \mainelementisotope{\arabic{element}} &
      \elconf{\arabic{element}}
      \stepcounter{element} \\
    }
    {}
\end{longtable}
\end{landscape}

\end{document}
